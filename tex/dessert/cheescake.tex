%% Mousse au Chocolat Example

\begin{otherlanguage}{ngerman}

\setHeadlines
{% translation
    inghead = Zutaten,
    prephead = Zubereitung,
    hinthead = Tipp,
    continuationhead = Fortsetzung,
    continuationfoot = Fortsetzung auf n\"achster Seite,
    portionvalue = Personen,
}

\begin{recipe}
[ % Optionale Eingaben
    preparationtime = {\unit[2]{Stunden}},
    portion = \portion{8},
    bakingtime = {\unit[1]{Stunde}},
    bakingtemperature={\protect\bakingtemperature{
        topbottomheat=\unit[190]{\textcelcius}
        }},
    portion = {\portion{8-10}},
    source = Erna
]
{K\"asekuchen}
    \graph
    {% Bilder
        % small=pic/glass,    % kleines Bild
        % big=pic/ingredients % großes (längeres) Bild
    }
    \ingredients
    {% Zutaten
        % 2 Tafeln & dunkle Schokolade (über \unit[70]{\%})\\
        % 3 & Eier\\
        % \unit[200]{ml} & Sahne\\
        % \unit[40]{g} & Zucker\\
        % \unit[50]{g} & Butter
        \unit[350]{g} & Zucker \\
        \unit[200]{g} & Magarine \\
        \unit[6] & Ei \\
        \unit[250]{g} &  Mehl \\
        \unit[250]{ml} &  Sonnenblumen\"ol \\
        \unit[750]{g} & Quark \\
        \unit[5]{EL} & Milch \\
        \unit[1]{} & Zitrone \\
        \unit[1]{} & Vanillepudding- Pulver \\
        \unit[1]{} & Vanillezucker \\
    }

    \preparation{%
    \step Zunächst wird der Boden zubereiten. Die vorgesehe- ne Menge reicht für 3 Böden aus. Es werden 100g
Zucker mit 200g Magarine schaumig gerührt.
    \step 1 Ei hinzufügen und 250g Mehl unterrühren, sodass eine feste Teigmasse entsteht und den Teig mindes-
tens eine Stunde im Kühlschrank ruhen lassen.
    \step In einer weiteren Schüssel werden 750g Quark und 5 Eier verrührt.
    \step Die restlichen 250g Zucker werden nach und nach hin- zugegeben. Nun das Öl langsam hinzugeben.
    \step Den Saft einer Zitrone, die Milch, das Puddingpulver, den Vanillezucker, sowie eine Prise Salz hinzugeben
und den Käse glattrühren.
    \step Den Boden in eine gut eingefettete Springform geben und den Käse daraufgeben.
    \step Den Kuchen eine Stunde im Ofen bei \unit[190]{\textcelcius} backen.

    }

    \hint
    {% Tipp
        Falls der Kuchen zu dunkel wird, kann man ihn mit Alufolie abdecken.
    }

\end{recipe}

\end{otherlanguage}
