%% Mousse au Chocolat Example

\begin{otherlanguage}{ngerman}

\setHeadlines
{% translation
    inghead = Zutaten,
    prephead = Zubereitung,
    hinthead = Tipp,
    continuationhead = Fortsetzung,
    continuationfoot = Fortsetzung auf n\"achster Seite,
    portionvalue = Personen,
}

\begin{recipe}
[ % Optionale Eingaben
    preparationtime = {\unit[20]{Minuten}},
    % bakingtime = {\unit[1]{Stunde}},
    % bakingtemperature={\protect\bakingtemperature{
    %     topbottomheat=\unit[190]{°C}
    %     }},
    portion = {\portion{2}},
    source = Stefan Gadatsch
]
{Ingwer-Basilikum Ramen}
    \graph
    {% Bilder
        % small=pic/glass,    % kleines Bild
        % big=pic/ingredients % großes (längeres) Bild
    }
    \ingredients
    {% Zutaten
        \unit[3]{Bund} & Fr\"uhlingszwiebeln \\
        \unit[1]{Zehe} & Knoblauch \\
        \unit[1]{kl. Knolle} & Ingwer \\
        & Basilikum \\
        & pflanzliches \"Ol \\
        \unit[1]{EL} & Chili\"ol \\
        \unit[1]{EL} & Sesam\"ol \\
        \unit[1]{EL} & Sojaso{\ss}e \\
        \unit[1]{EL} & Rei{\ss}essig \\
        \unit[2]{Packungen} & Instant Ramen
    }

    \preparation{%
      \step Fr\"uhlingszwiebeln, Ingwer und Knoblauch klein schneiden. Anschlie{\ss}end in einer Pfanne mit dem hei{\ss}en Pflanzen\"ol anschwitzen.

      \step F\"ur die So{\ss}e Chili\"ol, Sesam\"ol, Sojas{\ss}e und Rei{\ss}essig in einer Sch\"ussel vermischen.

      \step Instant Ramen nach Anleitung kochen und in die Pfanne mit den Fr\"uhlingszwiebeln und dem Knoblauch zugeben. Ebenfalls die gemengte So{\ss}e zugeben und alles gut verr\"uhren.

      \step In der Zwischenzeit 2 Eier 5 Minuten kochen und anschlie{\ss}end von der Schale befreien.

      \step Den Ramen auf einem Teller anrichten und das gepellte Ei darauf geben und mit Basilikum garnieren.
    }

    \hint
    {% Tipp
    }

\end{recipe}

\end{otherlanguage}
