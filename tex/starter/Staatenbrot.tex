% Complete recipe example
\begin{recipe}
[% 
    preparationtime = {\unit[15]{min}},
    bakingtime={\unit[30]{min}},
    bakingtemperature={\protect\bakingtemperature{
        % fanoven=\unit[230]{\textcelcius},
        topbottomheat=\unit[180]{\degree C},
        % topheat=\unit[195]{°C},
        % gasstove=Level 2
        }},
    portion = {\portion{6}},
    % calory={\unit[3]{kJ}},
    source = {Julias Mutter}
]
{Staatenbrot}
    
    \ingredients{%
        \unit[150]{g} & Haferflocken (kernig und zart)\\
        \unit[130]{g} & Sonnenblumenkerne\\
        \unit[60]{g}  & gehackte N\"usse\\
        \unit[50]{g}  & Leinsamen\\
        \unit[50]{g}  & Leinsamen geschrotet\\
        \unit[20]{g}  & K\"urbiskerne\\
        \unit[20]{g}  & Chiasamen\\
        \unit[20]{g}  & Flohsamenschalen\\
        \unit[1]{TL} & Salz\\
        \unit[3]{EL} & \"Ol\\
        \unit[2]{EL} & Algavendicksaft oder Honig oder Dattelsirup oder...\\
        \unit[350-360]{ml} & warmes Wasser\\
    }
    
    \preparation{%
        \step Alles verr\"uhren und in die Kastenform fest dr\"ucken  (ich mache das mehrmals). 12 Std stehen lassen und dann...
        \step Bei 190 Grad Ober-Unterhitze  30 Minuten backen,  dann aus der Form kippen (Backpapier) und nochmal ohne Form  30 Min backen.  
        \step Vorsichtig anschneiden (kalt)  oder in Scheiben einfrieren.  H\"alt lt. Rezept auch 7 Tage frisch

    }
    
    \hint{%
    \begin{itemize}
        \item La yema es indispensable para que queden melcochudos.
        \item Dejar enfriar antes de cortar.
    \end{itemize}
    }
    
\end{recipe}